% \documentclass{article}
\documentclass[]{acmart}
% \documentclass[sigplan,screen]{acmart}

% \usepackage[utf8]{inputenc}

\title{Type-guided synthesis for dynamic languages}
\author{Thomas Logan}
\date{October 2022}

\begin{document}

\maketitle

\section{Introduction}
\subsection*{Background}
Some of the most popular programming languages today are 
dynamic programming languages, which includes Javascript and Python. 
Javascript is become the language of the web, while Python the most popular
for data science and machine learning projects. 

Dynamic programming languages can make writing programs quick and easy 
because they don't require specifying static bounds on behavior.
The cost of this efficiency it may be impossible to statically decide 
if a program is has errors.  

Although dynamic languages already offer programming ease and efficiency, it
may be possible to increase the ease of writing program by autocompletion,
or synthesis of programs from surrounding context. 

\subsection*{Challenge}
This article presents a theoretical system that synthesizes terms from context 
in a dynamic language. In keeping with the spirit of dynamic languages, 
type annotations are not required, but they are optional. 

\subsection*{Evaluation}





% -- introduction
% /-

% - types have become the lingua franca of formal specification
%   - examples, propositions, abstract values, etc. may be encoded in the language of types
%   - type are merely phrases that may be more abstract than terms 

% - types have long been used to verify correct behavior of programs.
% - More recent work has demonstrated the utility of types to represent goals in 
%   - (dependently-typed) theorem proving
%   - program synthesis with refinement types

% - In using types as goals, the key idea is to propagate types down the syntax tree for decomposition
%   - providing smaller goals for local proofs/programs. 

% - a dynamic language is one where terms are not required to have static restrictions  
%   - from a static view, it is unityped. Every term may belong to the same type


% - type inference
%   - ML-like languages
%     - type inference is sound. 
%       - a term is instrinsicly associated with a static countable/inductive representation to type check 
%       - a type that is broader than the term but still finitely representable may be inferred

%   - dynamic languages
%     - type inference is unsound 
%       - idiomatic representation of term should be accepted without enumerating all cases   
%       - a term is not intrinsicly associated with some finitely representable/countable type
%       - sound inference would thus be overly restrictive  

% -/


% -- solution
% /-

% - what kind of terms can express useful behavior 
%   - function, record, variant
%   - 
%   - fix 
% - what kind of types can tightly bound runtime behavior 
%   - function type
%   - inductive function type, i.e. (indexed record) 
%     - related to Π types in dependent type theory
%   - record type
%   - variants type
%   - inductive variants type 
%   - inductive record of variants type, i.e. "relational type", i.e.(indexed variants) 
%     - one column indexes the other column
%     - related to Σ types in dependent type theory

% - how do we produce useful types to guide synthesis 
%   - type flow
%     - upward: when do we compose actual types and pop up
%       - we must flow up for all rules
%       - because we aren't guranteed any annotations 
%     - downward: when do we push down expected types and decompose
%       - we must flow down for all rules
%       - because we aren't guranteed complete terms
%   - type adaptation
%     - widening: infer types from arguments
%       - maintain leniency of expected type while 
%         - recording previously seen types (ty | ?)
%         - bounding actual types (ty | ?)
%     - narrowing: infer types from parameter types
%       - maintain leniency of actual type while 
%         - recording previously seen types (ty & ?)
%         - bounding expected types (ty & ?)

% - how do we leverage types to synthesize programs 
%   - leverage learn-to-syntheize tools
%   - sequent-calc symbolic search


% \begin{enumerate}
% \item hello everybody
% \end{enumerate}

\end{document}