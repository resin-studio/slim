\documentclass[letterpaper]{llncs}
\usepackage[letterpaper, margin=1.5in]{geometry}

\usepackage{mathpartir}
\usepackage{hyperref}
\usepackage{mathtools}
\usepackage{amsmath}
\usepackage{nccmath}
\usepackage{stmaryrd}
\usepackage{amssymb}
\usepackage{listings}


\usepackage{graphicx}
\graphicspath{ {./images/} }

\usepackage{url}

\makeatletter % allow us to mention @-commands
\def\arcr{\@arraycr}
\makeatother

\lstset{
    % identifierstyle=\color{violet},
    % textcolor=blue,
    % keywordstyle=\color{blue},
    keywordstyle=\text,
    basicstyle=\ttfamily,
    mathescape=true,
    showspaces=false,
    morekeywords={let, fix, in}
}
\usepackage[utf8]{inputenc}
% \usepackage[T1]{fontenc}


\title{Relational Types}
% \author{}
% \institute{}

\begin{document}
\maketitle


\section{Introduction}

\textbf{Context}. \newline
\textbf{Gap}. \newline
\textbf{Innovation}. \newline

\section{Overview}

\subsection{Language of types}
\textbf{Polymorphic}.  \newline
\textbf{Set-theoretic}. Union implies extrinsic/non-intrinsic. Implies Subtyping  \newline
\textbf{Relational}.  \newline

\subsection{Automation}
\textbf{Let-poly type-inference}. \newline
\textbf{Adjustments with union and intersection}. \newline
\textbf{Type checking two relational types}. \newline
\textbf{Path sensitivity}. \newline

\subsection{Examples}

\section{Related Techniques}

\textbf{HM type inference}.  \newline
\textbf{model-based subtyping}. Exemplified by XDuce and CDuce. complete subtyping.
% set theoretic notes: https://pnwamk.github.io/sst-tutorial/#%28part._sec~3asemantic-subtyping%29
% semantic subtyping: https://www.irif.fr/~gc/papers/semantic_subtyping.pdf
The terminology "semantic subtyping" vs "syntactic subtyping" are confusing terms. 
They actually refer to the meta-linguistic semantics of denotation 
meta-linguistic syntax of inference rules.
Better terms would be "model-based" vs "rule-based", or "denotational" vs "operational".
The former merely uses indirection into the language of sets.
Both are semantics that are defined in terms of syntax, either directly or indirectly.
Really, the dichotomy should be referred to as "indirect subtyping" vs "direct subtyping".
All semantics are defined in terms of syntax, either directly or indirectly. 
necessary for a complete semantics at the cost of undecidability; syntactic inference rules are incomplete, but decidable.
This basically seems like kicking the can down the road. Eventually there will need to be 
a syntactic-based procedure for checking inhabitation of sets.
\newline
\textbf{Typescript}. Unsound. Maybe not as lenient?  \newline
\textbf{Refinement Types}. a restricted to intersection; implies intrinsic top\newline
\textbf{Predicate Subtyping}. An extension of refinement types exemplified by Liquid Types.
RLT starts with an invalid post-condition, then weakens return type to the (strongest) valid post-condition from outside in by expanding unions. 
RLT starts with an invalid pre-condition, then strengthens parameter type to a (weakest) valid pre-condition from inside out by adding intersections. 
Liquid types starts with an invalid post-condition, then uses iterative weakening by dropping conjunctions until a valid post-condition is reached.
\newline
\textbf{Abstraction Refinement}. Type unification over subtyping resembles abstraction refinement  
where solving for variables on different sides of the subtyping relation corresponds to
solving with the abstractor vs solving with the refiner.
\newline
\textbf{Craig interpolation}. extracting an inductive type with unions and intersections 
from a recursive program without needing to specify a predicate universe might be similar to
craig interpolation. \newline
\textbf{PDR}. exemplified by IC3. RLT infers abstract type for return type, 
then safely constrains the variables in previous step (fix's antecedent) to subtype the least fixed point.
This, in essence, propagates the type for the last step to the previous steps.
This is safe as antecedent is stronger than consequent at any step.
the technique in PDR for propagating loss points might be related. 
 \newline 




\end{document}


